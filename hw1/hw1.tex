\documentclass{article}
\usepackage[T1]{fontenc}
\usepackage{amssymb}
\usepackage{listings}
\usepackage[utf8x]{inputenc}
\usepackage{amsmath,amssymb}
\usepackage[russian,english]{babel}
\lstset{
    language=Octave,
    frame=single
}
\usepackage[a4paper,top=1cm,bottom=2cm,left=1.5cm,right=1cm,marginparwidth=1.75cm]{geometry}

\makeatletter
\def\@seccntformat#1{
  \expandafter\ifx\csname c@#1\endcsname\c@section\else
  \csname the#1\endcsname\quad
  \fi}
\makeatother

\begin{document}

\selectlanguage{russian}
\title{Домашнее задание 1, ТВМС}
\author{Ковешников Глеб, М3238}
\maketitle
\begin{center}
    Метод Монте-Карло \\
    Вариант №10
\end{center}

\begin{quote}
\selectlanguage{russian}
\section{1 задание}
\subsection{Формулировка}
        Методом Монте-Карло оценить объем части тела \{$F(\tilde x) \leq c$\}, заключенной в $k$-мерном кубе с ребром $[0, 1]$. 
        Функция имеет вид $F(\tilde x) = f(x_1) + f(x_2) + ... + f(x_k)$.
        Для выбранной надежности $\gamma \geq 0.95$ указать асимптотическую точность оценки и построить асимптотический доверительный интервал для истинного значения объёма. \\
        Используя объём выборки $n = 10^4$ и $n = 10^6$, оценить скорость сходимости и показать, что доверительные интервалы пересекаются.
\subsection{Входные данные}
        \begin{itemize}
            \item Функция имеет вид $f(x) = x^a$
            \item Куб размерностью $k = 10$
            \item Параметр $c = 2.21$ 
	    \item Параметр $a = 3$
        \end{itemize}
\subsection{Исходный код программы}
        \begin{minipage}{\linewidth}
            \lstinputlisting{v1.m}
        \end{minipage}
\subsection{Выходные данные}
	\begin{minipage}{\linewidth}
            \lstinputlisting{res1.txt}
        \end{minipage}
\subsection{Вывод}
        Доверительный интервал при $n = 10^6$ содержится в интервале при $n = 10^4$.\\
        При увеличении числа итераций в $100$ раз ширина доверительного интервала уменьшилось в $10$ раз.
\section{2 задание}
\subsection{Формулировка}
    Построить оценку интегралов (представить интеграл как математическое ожидание функции,
    зависящей от случайной величины с известной плотностью) и для выбранной надежности $\gamma \geq 0.95$ указать
    асимптотическую точность оценки и построить асимптотический доверительный интервал для истинного
    значения интеграла. 
\subsection{Интеграл 1}
	${\displaystyle \int\limits_{-\infty}^{\infty} \sqrt{1 + x^2} exp \left( \dfrac{-(x + 2)^2}{4} \right) dx}$
	\subsubsection{Исходный код программы}
        	\begin{minipage}{\linewidth}
        	    \lstinputlisting{v2.m}
        	\end{minipage}
	\subsubsection{Выходные данные}
		\begin{minipage}{\linewidth}
        	    \lstinputlisting{res2.txt}
        	\end{minipage}
	\subsubsection{Вывод}
		Истинное значение интеграла содержится в доверительном интервале при $n = 10^4$ и $n = 10^6$. Значение, полученное методом Монте-Карло отличается от значения, полученного методом $quad$, на $7.5 \cdot 10^{-4}$.\\
        	При увеличении числа итераций в $100$ раз, ширина доверительного интервала уменьшилось в $10$ раз.
\subsection{Интеграл 2}
	${\displaystyle \int\limits_{0}^{5} \dfrac{\sin (x)}{x^2+1} dx}$
	\subsubsection{Исходный код программы}
        	\begin{minipage}{\linewidth}
        	    \lstinputlisting{v3.m}
        	\end{minipage}
	\subsubsection{Выходные данные}
		\begin{minipage}{\linewidth}
        	    \lstinputlisting{res3.txt}
        	\end{minipage}
	\subsubsection{Вывод}
        	Истинное значение интеграла содержится в доверительном интервале при $n = 10^4$ и $n = 10^6$. Значение, полученное методом Монте-Карло отличается от значения, полученного методом $quad$, на $1.6 \cdot 10^{-4}$.\\
        	При увеличении числа итераций в $100$ раз, ширина доверительного интервала уменьшилось в $10$ раз.
\end{quote}
\end{document}
