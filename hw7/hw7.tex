\documentclass{article}
\usepackage[T1]{fontenc}
\usepackage{amssymb}
\usepackage{listings}
\usepackage[utf8x]{inputenc}
\usepackage{amsmath,amssymb}
\usepackage{pdfpages}
\usepackage[russian,english]{babel}
\lstset{
    language=Octave,
    frame=single,
    inputencoding=utf8x,
    extendedchars=\true,
    texcl=\false,
    breaklines=true,
    breakatwhitespace=true,
    commentstyle={}
}
\usepackage[a4paper,top=1cm,bottom=2cm,left=1.5cm,right=1cm,marginparwidth=1.75cm]{geometry}

\makeatletter
\def\@seccntformat#1{
  \expandafter\ifx\csname c@#1\endcsname\c@section\else
  \csname the#1\endcsname\quad
  \fi}
\makeatother

\begin{document}

\selectlanguage{russian}
\title{ТВМС, Лабораторная работа 7, Вариант 10}
\author{
	Ковешников Глеб, M3238\\
	kovg16@gmail.com
}
\maketitle

\begin{quote}
\selectlanguage{russian}
\section{Формулировка}
	\large Для случайной величины $X \thicksim P(\lambda)$, гипотезы $H_0:\lambda = \lambda_0 = 4$, альтернативы $H_1:\lambda > \lambda_0$, при $n = 400$ и $\overline{X_n} = 4.7$ построить доверительный интервал для $\alpha = 0.1$ и проверить гипотезу для $\gamma = 0.95$.
\section{Входные данные}
	\Large
        \begin{itemize}
		\item $t_{1 - \alpha} = t_{0.9} = 1.65$
		\item $c_{\gamma} = c_{0.95} = 1.65$
		\item $I(\lambda) = \frac{1}{\lambda} \Rightarrow I(\overline{\lambda}) = \frac{1}{4.7}$
        \end{itemize}
	\large
\section{Доверительный интервал}	
        Построим доверительный интервал: \\ \\
	\Large
	$I = \Biggl[\overline{\theta_n} - \frac{t_{1 - \alpha}}{\sqrt{nI(\theta_n)}} ; \overline{\theta_n} + \frac{t_{1 - \alpha}}{\sqrt{nI(\theta_n)}}\Biggr] = \Biggl[4.7 - \frac{1.65}{\sqrt{\frac{400}{4.7}}} ; 4.7 + \frac{1.65}{\sqrt{\frac{400}{4.7}}}\Biggr] = \bigl[4.52 ; 4.88\bigr]$ \\ \\ \\
	$\lambda_0 \notin I$ \\ \\
	\large
	Таким образом, гипотеза отвергается.
\subsection{Проверка гипотезы}
	Воспользуемся критерием для правосторонней альтернативы: \\
	\Large
	\begin{equation*}
		\psi^{*}_{n, \alpha} = 
 	\begin{cases}
		1\text{, $\sqrt{nI(\theta_0)} \cdot (\overline{\theta_n} - \theta_0) \ge c_{\gamma}$}\\
		0\text{, $\sqrt{nI(\theta_0)} \cdot (\overline{\theta_n} - \theta_0) < c_{\gamma}$}
 	\end{cases}
	\end{equation*}
	$\psi^{*}_{n, \alpha} = \bigl[\sqrt{nI(\theta_0)} \cdot (\overline{\theta_n} - \theta_0) \ge c_{\gamma}\bigr] = \bigl[6.46 \ge 1.65\bigr] = 1$ \\ \\
	\large
	Таким образом, гипотеза снова отвергается.
\end{quote}
\end{document}
